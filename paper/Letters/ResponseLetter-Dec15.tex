\documentclass[a4paper]{letter}
\pagenumbering{gobble}

%----------------------------------------------------------------------------------------
%	DOCUMENT MARGINS
%----------------------------------------------------------------------------------------

\textwidth 6.75in
\textheight 9.25in
\oddsidemargin -.25in
\evensidemargin -.25in
\topmargin -1in
\longindentation 0.50\textwidth
\parindent 0.4in


\signature{Marisa Casillas \\ marisa.casillas$@$mpi.nl}
\address{Wundtlaan 1 \\ 6525 XD Nijmegen \\ The Netherlands}

\begin{document}
\begin{letter}{}

\opening{Dear Profs. Dr. Pickering and Dr. Gerrig,}

We resubmit our manuscript on ``The development of children's ability to track and predict turn structure in conversation'' to the \textit{Journal of Memory and Language}. We apologize for our delay in getting the manuscript resubmitted; it took quite a long time to implement one of the suggestions from Reviewer 3, mostly because I had to learn quite a bit of new technical skill to carry his or her suggested analysis out on our computation-heavy data set. In addition to that, we have included a two other follow-up analyses in the Supplementary materials, both inspired by the other reviewers' comments. We have tried to make the text more succinct and the results and discussion sections more distinct and straightforward, following requests from all three reviewers. We hope that, with these changes, we have greatly improved the paper's clarity. Please find a description of specific comments addressed below. Thank you very much for your reconsideration. Please let us know if there is any further information that you require about the submission.

\textbf{Manuscript identifying details:}

\begin{tabular}{ll}
\textit{ID:} & JML-14-235 \\
\textit{Title:} & The development of children's ability to track and \\
& predict turn structure in conversation \\
\textit{Authors:} & Marisa Casillas\textsuperscript{a} and Michael C. Frank\textsuperscript{b} \\
\textit{Affiliations:} & a) Max Planck Institute for Psycholinguistics and \\
& b) Stanford University
\end{tabular}

\closing{Sincerely,}

We thank all three reviewers for their thorough and helpful comments on the previously submitted manuscript. We have tried to address all of the reviewers' comments, within the manuscript and/or here in the response letter.

\textbf{Reviewer 1}

\textit{1. 333ms may be too long for older children's saccadic planning time and may therefore inflate the anticipatory looking rates for older children.}

We were concerned about this, too, but we unfortunately did not collect saccadic planning norms for children ages 1--6. We did not feel comfortable (semi-)arbitrarily assigning saccadic planning times to different ages, so in the newly analyzed data we take the conservative approach of using adult-like planning times (200ms) for children at all ages in both experiments.

\smallskip

\textit{2. The description of the random baseline permutations was not clear.}

We have tried our best to clarify this description (Section XX) and have added specific pieces of information where requested.

\smallskip

\textit{3. Are anticipatory gaze shifts driven by boredom (and not by anticipation)? Do children generally just look away as the speaker goes on?}

This alternative hypothesis makes a valuable test of what we measured in the experiments. In the supplementary materials you can now find a graph comparing theoretically boredom-driven lookers and our real lookers. We modeled the proportion of participants looking at the current speaker for our real data and for (fake) boredom-driven lookers (who look away from the current speaker at a constant rate, starting 1 second after the onset of speech) across turns of different length. Even in short turns, where there is very little temporal room for children to behave differently from the boredom-driven lookers, there is a clear difference between the boredom-driven and the real data. (Figure XX). We take this to mean that children are unlikely to be looking away simply due to boredom.

\smallskip

\textit{4. The reported results form the models models are hard to follow.}

Thank you for pointing this out! We have added a little more information about how each variable was coded and have put summary tables of every model's output into the text.

\smallskip

\textit{5. ``No speech'' should be the reference level against which the linguistic conditions are compared in Experiment 2.}

This is a very useful comment, thanks! We agree and have changed our analyses to accommodate it. We had started with the ``normal speech'' condition as the reference level because that is what was done in related work on adult turn-end prediction. In the end, however, we too preferred the additive model, and have changed the text accordingly.

\newpage

\textit{6. The children appear to not use linguistic information at all.}

XX

\smallskip

\textit{7. It is too hard to link the conclusions in the discussion back to the statistical results and the effects themselves are not talked about in a clear enough way.}

We have tried our best to make this clearer, first by clarifying the statistical analyses and outputs, and second by trying to explicitly link each claim to an individual result (through the use of figure and table references). We hope that this will help readers better connect the results to our interpretations.

\smallskip

\textit{8. There are too many references to unpublished work.}

The single upside to taking so long in implementing these comments is that all of the unpublished work referred to previously is now published or in press!

\smallskip

\textit{Small miscellany has been addressed directly in the text.}

\bigskip

\textbf{Reviewer 2}

\textit{1. It is not clear which claims are backed up by which analyses. Analyses should be carried out for each group separately.}

Thanks! We have tried to improve the linkage between our claims and analyses. Please see responses 4 and 7 under Reviewer 1 for a little more detail. The main focus of our analyses was the effect that age (as a continuous variable) has on anticipatory gaze. However there are two cases in which our main analyses are not satisfying without further, age-group analyses: (1) age sometimes showed itself to interact with other factors (e.g., age and language condition), and (2) without further tests we can not say whether particular age groups themselves statistically differ from chance, across or within conditions. Following your suggestion, we have therefore done two things. First, we added follow-up two-tailed t-tests, making pairwise comparisons between age groups for significant interactions so that we could find out how the interacting factor (e.g., language condition) significantly changes across age groups (Sections XX). Second, we have added individual models of the youngest age groups (ages 1, 2, and 3), comparing their real looking behavior to our randomly permuted data across conditions (Section XX). These follow-up analyses, which are summarized in the text, are primarily described in the supplementary materials (Section XX).

\smallskip

\textbf{Reviewer 3}


\end{letter}
\end{document}